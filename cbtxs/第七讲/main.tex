% !Mode:: "TeX:UTF-8"
\documentclass[12pt,a4paper]{article}

%%%%%%%%------------------------------------------------------------------------
%%%% 日常所用宏包

%% 控制页边距
% 如果是beamer文档类, 则不用geometry
\makeatletter
\@ifclassloaded{beamer}{}{\usepackage[top=2.5cm, bottom=2.5cm, left=2.5cm, right=2.5cm]{geometry}}
\makeatother

%% 控制项目列表
\usepackage{enumerate}

%% 多栏显示
\usepackage{multicol}

%% 算法环境
\usepackage{algorithm}  
\usepackage{algorithmic} 
\usepackage{float} 

%% 网址引用
\usepackage{url}

%% 控制矩阵行距
\renewcommand\arraystretch{1.4}

%% hyperref宏包,生成可定位点击的超链接,并且会生成pdf书签
\makeatletter
\@ifclassloaded{beamer}{
\usepackage{hyperref}
\usepackage{ragged2e} % 对齐
}{
\usepackage[%
    pdfstartview=FitH,%
    CJKbookmarks=true,%
    bookmarks=true,%
    bookmarksnumbered=true,%
    bookmarksopen=true,%
    colorlinks=true,%
    citecolor=blue,%
    linkcolor=blue,%
    anchorcolor=green,%
    urlcolor=blue%
]{hyperref}
}
\makeatother



\makeatletter % 如果是 beamer 不需要下面两个包
\@ifclassloaded{beamer}{
\mode<presentation>
{
} 
}{
%% 控制标题
\usepackage{titlesec}
%% 控制目录
\usepackage{titletoc}
}
\makeatother

%% 控制表格样式
\usepackage{booktabs}

%% 控制字体大小
\usepackage{type1cm}

%% 首行缩进,用\noindent取消某段缩进
\usepackage{indentfirst}

%% 支持彩色文本、底色、文本框等
\usepackage{color,xcolor}

%% AMS LaTeX宏包: http://zzg34b.w3.c361.com/package/maths.htm#amssymb
\usepackage{amsmath,amssymb}
%% 多个图形并排
\usepackage{subfig}
%%%% 基本插图方法
%% 图形宏包
\usepackage{graphicx}
\newcommand{\red}[1]{\textcolor{red}{#1}}
\newcommand{\blue}[1]{\structure{#1}}
\newcommand{\brown}[1]{\textcolor{brown}{#1}}
\newcommand{\green}[1]{\textcolor{green}{#1}}


%%%% 基本插图方法结束

%%%% pgf/tikz绘图宏包设置
\usepackage{pgf,tikz}
\usetikzlibrary{shapes,automata,snakes,backgrounds,arrows}
\usetikzlibrary{mindmap}
%% 可以直接在latex文档中使用graphviz/dot语言,
%% 也可以用dot2tex工具将dot文件转换成tex文件再include进来
%% \usepackage[shell,pgf,outputdir={docgraphs/}]{dot2texi}
%%%% pgf/tikz设置结束


\makeatletter % 如果是 beamer 不需要下面两个包
\@ifclassloaded{beamer}{

}{
%%%% fancyhdr设置页眉页脚
%% 页眉页脚宏包
\usepackage{fancyhdr}
%% 页眉页脚风格
\pagestyle{plain}
}

%% 有时会出现\headheight too small的warning
\setlength{\headheight}{15pt}

%% 清空当前页眉页脚的默认设置
%\fancyhf{}
%%%% fancyhdr设置结束


\makeatletter % 对 beamer 要重新设置
\@ifclassloaded{beamer}{

}{
%%%% 设置listings宏包用来粘贴源代码
%% 方便粘贴源代码,部分代码高亮功能
\usepackage{listings}

%% 设置listings宏包的一些全局样式
%% 参考http://hi.baidu.com/shawpinlee/blog/item/9ec431cbae28e41cbe09e6e4.html
\lstset{
showstringspaces=false,              %% 设定是否显示代码之间的空格符号
numbers=left,                        %% 在左边显示行号
numberstyle=\tiny,                   %% 设定行号字体的大小
basicstyle=\footnotesize,                    %% 设定字体大小\tiny, \small, \Large等等
keywordstyle=\color{blue!70}, commentstyle=\color{red!50!green!50!blue!50},
                                     %% 关键字高亮
frame=shadowbox,                     %% 给代码加框
rulesepcolor=\color{red!20!green!20!blue!20},
escapechar=`,                        %% 中文逃逸字符,用于中英混排
xleftmargin=2em,xrightmargin=2em, aboveskip=1em,
breaklines,                          %% 这条命令可以让LaTeX自动将长的代码行换行排版
extendedchars=false                  %% 这一条命令可以解决代码跨页时,章节标题,页眉等汉字不显示的问题
}}
\makeatother
%%%% listings宏包设置结束


%%%% 附录设置
\makeatletter % 对 beamer 要重新设置
\@ifclassloaded{beamer}{

}{
\usepackage[title,titletoc,header]{appendix}
}
\makeatother
%%%% 附录设置结束


%%%% 日常宏包设置结束
%%%%%%%%------------------------------------------------------------------------


%%%%%%%%------------------------------------------------------------------------
%%%% 英文字体设置结束
%% 这里可以加入自己的英文字体设置
%%%%%%%%------------------------------------------------------------------------

%%%%%%%%------------------------------------------------------------------------
%%%% 设置常用字体字号,与MS Word相对应

%% 一号, 1.4倍行距
\newcommand{\yihao}{\fontsize{26pt}{36pt}\selectfont}
%% 二号, 1.25倍行距
\newcommand{\erhao}{\fontsize{22pt}{28pt}\selectfont}
%% 小二, 单倍行距
\newcommand{\xiaoer}{\fontsize{18pt}{18pt}\selectfont}
%% 三号, 1.5倍行距
\newcommand{\sanhao}{\fontsize{16pt}{24pt}\selectfont}
%% 小三, 1.5倍行距
\newcommand{\xiaosan}{\fontsize{15pt}{22pt}\selectfont}
%% 四号, 1.5倍行距
\newcommand{\sihao}{\fontsize{14pt}{21pt}\selectfont}
%% 半四, 1.5倍行距
\newcommand{\bansi}{\fontsize{13pt}{19.5pt}\selectfont}
%% 小四, 1.5倍行距
\newcommand{\xiaosi}{\fontsize{12pt}{18pt}\selectfont}
%% 大五, 单倍行距
\newcommand{\dawu}{\fontsize{11pt}{11pt}\selectfont}
%% 五号, 单倍行距
\newcommand{\wuhao}{\fontsize{10.5pt}{10.5pt}\selectfont}
%%%%%%%%------------------------------------------------------------------------


%% 设定段间距
\setlength{\parskip}{0.5\baselineskip}

%% 设定行距
\linespread{1}


%% 设定正文字体大小
% \renewcommand{\normalsize}{\sihao}

%制作水印
\RequirePackage{draftcopy}
\draftcopyName{XTUMESH}{100}
\draftcopySetGrey{0.90}
\draftcopyPageTransform{40 rotate}
\draftcopyPageX{350}
\draftcopyPageY{80}

%%%% 个性设置结束
%%%%%%%%------------------------------------------------------------------------


%%%%%%%%------------------------------------------------------------------------
%%%% bibtex设置

%% 设定参考文献显示风格
% 下面是几种常见的样式
% * plain: 按字母的顺序排列,比较次序为作者、年度和标题
% * unsrt: 样式同plain,只是按照引用的先后排序
% * alpha: 用作者名首字母+年份后两位作标号,以字母顺序排序
% * abbrv: 类似plain,将月份全拼改为缩写,更显紧凑
% * apalike: 美国心理学学会期刊样式, 引用样式 [Tailper and Zang, 2006]

\makeatletter
\@ifclassloaded{beamer}{
\bibliographystyle{apalike}
}{
\bibliographystyle{unsrt}
}
\makeatother


%%%% bibtex设置结束
%%%%%%%%------------------------------------------------------------------------

\input{xecjk_preamble.tex}
%\linespread{1.5}
\usepackage{parskip}%首行缩进
\setlength{\parindent}{0cm}%首行缩进

\title{}
\author{作者}
\date{\chntoday}
\begin{document}
\maketitle
\newpage
\section*{\color{blue}第七讲\quad 子空间迭代方法}
\subsection*{基本思想}
在一个{\color{blue}维数较低的子空间}中寻找解析解的一个{\color{blue}最佳近似}.子空间迭代算法的主要过程可以分解为下面三步:
\begin{enumerate}[(1)]
\item 寻找合适的子空间;\\
\item 在该子空间中求“最佳近似”;\\
\item 若这个近似解满足精度要求,则停止计算;否则,重新构造一个新的子空间,并返回第(2)步.
\end{enumerate}
这里主要涉及到的{\color{blue}两个关键问题}是:
\begin{enumerate}
\item 如果选择和更新子空间;
\item 如何在给定的子空间中寻找“最佳近似”.
\end{enumerate}
关于第一个问题,目前较成功的解决方案就是使用{\color{blue}Krylov子空间}.\\
\section{\color{blue}Krylov子空间}
设$A \in \mathbb{R}^{n \times n}, r \in \mathbb{R}^{n}$,则由$A$和$r$生成的$m$维{\color{blue}Krylov子空间}定义为
$$
\boxed{\mathcal{K}_{m}=\mathcal{K}_{m}(A, r) \triangleq \operatorname{span}\left\{r, A r, A^{2} r, \ldots, A^{m-1} r\right\}, \quad m \leq n}
$$
设$\operatorname{dim} \mathcal{K}_{m}=m$,令$v_{1}, v_{2}, \ldots, v_{m}$是$\mathcal{K}_{m}$的一组基,则$\forall x \in \mathcal{K}_{m}$可表示为
$$x=y_{1} v_{1}+y_{2} v_{2}+\cdots+y_{m} v_{m} \triangleq V_{m} y$$
{\color{blue}寻找“最佳近似”$x^{(m)}$}转化为
\begin{enumerate}
\item 寻找一组合适的基$v_{1}, v_{2}, \ldots, v_{m}$;
\item 求出$x^{(m)}$在这组基下面的表出系数$y^{(m)}$.
\end{enumerate}
\subsection*{基的选取: Arnoldi过程}
最简单的基:$\left\{r, A r, A^{2} r, \ldots, A^{m-1} r\right\} \longmapsto$非正交,稳定性得不到保证.\\
{\color{blue}Arnoldi过程}:将$\left\{r, A r, A^{2} r, \ldots, A^{m-1} r\right\}$单位正交化\\
\begin{tabular}{l}
1:$v_{1}=r /\|r\|_{2}$\\
2:for $j=1$ \text { to } $m$ do\\
3:\qquad $z=A v_{j}$\\
4:\qquad for $i = 1$ to $j$ do \quad{\color{red}\% MGS正交化过程}\\
5:\qquad \qquad $h_{i, j}=\left(v_{i}, z\right), \quad z=z-h_{i, j} v_{i}$\\
6:\qquad end for\\
7:\qquad $h_{j+1, j}=\|z\|_{2}$\quad{\color{red}\% if $h_{j+1, j}=0$break, endif}\\
8:$v_{j+1}=z / h_{j+1, j}$\\
9:end for\\
\end{tabular}
\subsection*{Arnoldi过程的矩阵表示}
记$V_{m}=\left[v_{1}, v_{2}, \ldots, v_{m}\right]$
$$
H_{m+1, m}=\left[\begin{array}{ccccc}{h_{1,1}} & {h_{1,2}} & {h_{1,3}} & {\cdots} & {h_{1, m}} \\
 {h_{2,1}} & {h_{2,2}} & {h_{2,3}} & {\cdots} & {h_{2, m}} \\ 
 {} & {h_{3,2}} & {h_{3,3}} & {\cdots} & {h_{3, m}} \\
  {} & {}&{\ddots} & {\ddots} & {\vdots} \\
   {} & {} &{}& {h_{m, m-1}} & {h_{m, m}} \\
    {} & {} &{}& {h_{m+1, m}}
    \end{array}\right] \in \mathbb{R}^{(m+1) \times m}
$$
则由Arnoldi过程可知
$$
A v_{j}=h_{1, j} v_{1}+h_{2, j} v_{2}+\cdots+h_{j, j} v_{j}+h_{j+1, j} v_{j+1}
$$
所以有
\begin{align*}
A V_{m}=V_{m+1} H_{m+1, m}=V_{m} H_{m}+h_{m+1, m} v_{m+1} e_{m}^{T}
\tag{7.1}
\end{align*}
其中$H_{m}=H_{m+1, m}(1 : m, 1 : m), e_{m}=[0, \ldots, 0,1]^{T} \in \mathbb{R}^{m}$.
由于$V_{m}$是列正交矩阵,上式两边同乘$V_{m}^{T}$可得
\begin{align*}
V_{m}^{T} A V_{m}=H_{m}
\tag{7.2}
\end{align*}
等式(7.1)和(7.2)是Arnoldi过程的两个重要性质.\\
Lanczos过程\\
若$A$对称,则$H_m$为对称三对角,记为$T_m$,即
\begin{align*}
T_{m}=\left[\begin{array}{cccc}{\alpha_{1}} & {\beta_{1}} & {} & {} \\ {\beta_{1}} & {\ddots} & {\ddots} & {} \\ {} & {\ddots} & {\ddots} & {\beta_{m-1}} \\ {} & {} & {\beta_{m-1}} & {\alpha_{m}}\end{array}\right]
\tag{7.3}
\end{align*}
Lanczos过程的性质与{\color{blue}三项递推公式}(令$v_{0}=0$和$\beta_{0}=0$)
\begin{array}{lll}
${A V_{m}=V_{m} T_{m}+\beta_{m} v_{m+1} e_{m}^{T}}$&${}$&$\tag{7.4}$\\ 
${V_{m}^{T} A V_{m}=T_{m}}$&${}$&$\tag{7.5}$
\end{array}
$$
\beta_{j} v_{j+1}=A v_{j}-\alpha_{j} v_{j}-\beta_{j-1} v_{j-1}, \quad j=1,2, \ldots
$$
{\color{blue}Lanczos过程}
\begin{tabular}{l}
1:Set$v_0=0$and$$\beta_{0}=0$\\
2:$v_{1}=r /\|r\|_{2}$\\
3:for $j=1$ \text { to } $m$ do\\
4:\qquad $z=A v_{j}$\\
5:\qquad $\alpha_{j}=\left(v_{j}, z\right)$\\
6:\qquad $z=z-\alpha_{j} v_{j}-\beta_{j-1} v_{j-1}$\\
7:\qquad $\beta_{j}=\|z\|_{2}$\\
8:\qquad if $\beta_(j)=0$then break ,end if\\
9:\qquad $v_{j+1}=z / \beta_j$\\
10:end for\\
\end{tabular}






















































































































































































































































\cite{tam19912d}
\bibliography{ref}
\end{document}
